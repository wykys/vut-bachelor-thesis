\chapter*{Úvod}
\phantomsection
\addcontentsline{toc}{chapter}{Úvod}

Laser Game, někdy také nazývána i Laser Tag, je označení pro společenské hry, motivované sci--fi, využívající moderní elektroniku. Hráči získávají body za zásahy \zk{IR} paprskem z jejich zbraně do zásahových čidel soupeřů, či jiných oběktů zásahovým čidlem vybaveným. A naopak o své body přicházejí pokud se jejich zásahová čidla stanou terčem nepřátelských \zk{IR} paprsků.

Nejčasteji se hra odehrává v místnosi speciálně upravené pro tento účel, označované jako aréna. Stěny, strop a podlaha arény bývají zpravyhla pokryty tmavou barvou (černé koberce, koženky a podobné materiály) tak, aby se zabránilo odrazům \zk{IR} paprsků. Zdi v arénách jsou často vypolstrovány, aby se snížilo riziko zranění hrůčů. Obvykle arény nejsou jen klasické místnosti ze čtyřmi stěnami a rovnou podlahou, nýbrž často bývají značně členité, vybavené úkryty a překážkamy.

Pro spestření hry mohou být arény vybaveny i další elektornikou, například optickými závoramy, miny či světělnými a kouřovými efekty.
