\chapter*{Úvod}
\phantomsection
\addcontentsline{toc}{chapter}{Úvod}

% bližší rozbor a diskuze zadání, jeho upřesnění a doplnění, konkretizace cílů práce

Práce se zabývá společenskou, sportovní, zábavnou hrou zvanou laser game. Nejprve se zaměřuje na vymezení pojmu hry laser game, následně přejde k její historii a vzniku. V teoretické části práce jsou diskutovány optoelektronické součástky, které je možné pro návrh systému laser game použít se zaměřením na jejich funkci.

Autorovým cílem je vyvinout herní systém, který bude možné provozovat i v komerčních arénách.

Stěžejní částí práce je návrh komunikačního rozhraní a protokolu pro předávání informací pomocí infračerveného záření. A protokolu pro rádiovou komunikaci mezi vestami a směrovačem.

% zdroj toho počtu?
Laser game arén existuje i v ČR velký počet. Převážná většina z nich používá vybavení, které se pořizuje za statisíce korun a přitom jsem názoru, že je reálné jej vyrobit mnohem levněji. Tato práce si klade za cíl vyvinout systém, který bude možné nasadit v arénách a jeho pořizovací cena bude přijatelnější.
