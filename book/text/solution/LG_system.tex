\section{Úvod do problematiky Laser Game systému}

\subsection{Laser Game}
\zkratka{LG}, někdy také nazývána i Laser Tag, je označení pro společenské hry, motivované sci-fi, využívající moderní elektroniku. Hráči získávají body za zásahy \zkratka{IR} paprskem z jejich zbraně do zásahových čidel soupeřů, či jiných objektů zásahovým čidlem vybaveným. A naopak o své body přicházejí, pokud se jejich zásahová čidla stanou terčem nepřátelských \zkratka{IR} paprsků. Cílem každého hráče je zíshat co nejvíce bodů.

Nejčasteji se hra odehrává v místnosi speciálně upravené pro tento účel, označované jako aréna. Stěny, strop a podlaha arény bývají zpravidla pokryty tmavou barvou (černé koberce, koženky a podobné materiály) tak, aby se zabránilo odrazům \zkratka{IR} paprsků. Zdi v arénách jsou často vypolstrovány, aby se snížilo riziko zranění hráčů. Obvykle arény nejsou jen klasické místnosti se čtyřmi stěnami a rovnou podlahou, nýbrž často bývají značně členité, vybavené úkryty a překážkami.

Pro zpestření hry mohou být arény vybaveny i další elektornikou, například optickými závorami, minami či světělnými a kouřovými efekty.



\subsection{Historie Laser Game}
%\footnote{https://en.wikipedia.org/wiki/Laser_tag  21.11.2017}
Koncem sedmdesátých a na začátku osmdesátách let vyvýjela firma Lockheed Martin Information Systems pro armádu spojených států amerických systém, využívající \zkratka{IR} paprsky pro tréning boje navývaný \zkratka{MILIES}. Systém funguje na stjeném principu jako \zkratka{LG}. Voják má na hlavni zbraně \zkratka{IR} vysílač. Ten reaguje na stisk spoutě. Voják je oděn ve speciálním obleku z \zkratka{IR} senzory. Při zásahu vojáka je odeslány z obleku informace do dekodéru, který vyhodtí, jestli je voják zraněný a nebo mrtev. Používání tohoto systému je bezpečnější než využívání klasických zbraní. V roce 1992 vyšla nová verze tohoto systému pojmenovaná MILIES2000, který navíc po zabití vojáka zablokuje jeho další střelbu a akusticky ohlásí zasažení. \zkratka{MILIES} využívá od roku 2002 i armáda žeské rebubliky. Tento systém používají i ostatní země \zkratka{NATO}.


\subsection{Laser Game systém}
\zkratka{LG} systém je soustava zahrnující \zkratka{HW}, \zkratka{FW} a \zkratka{SW} umožnující hrát \zkratka{LG}. \zkratka{LG} systém lze tydy chápat jako vybavení nezbytné pro provozování \zkratka{LG} arény.

Nejdůležitějšími součástmi \zkratka{LG} systému jsou vesty, zbraně, kominikační router a řídící \zkratka{PC}, \zkratka{LG} systém může být dále doplněn například o optické závory, miny či bomby.

\subsection{Řídící počítač}
Řídící počítač má za úkol prostřednictvím routru komunikovat s vestami, případně i dalšími zařízeními v síti. Hostí řídící \zkratka{SW}, pomocí nehož operátor arény ovládá hru. Zajišťuje nakonfigurování vest. V průběhu hry stahuje z vest údaje a vyhodnocuje pořadí hráčů.

\subsection{Router}
Zajišťuje směrování v bezdrátové \zkratka{LG} síti. Sprostředkovává tedy přenos dat z vest do počítače a také z počítače do vest.

\subsection{Vesta}
Vesta je oděv který nosí hráči \zkratka{LG}. Má dvě hlavní funkce, detekovat zásahy a komunikovat s řídícím počítačem. Další funkcí vesty je barevná identifikace hráčů pomocí \zkratka{RGB} \zkratka{LED}. Vesta může zajišťovat i zvukové efekty, zobrazovat hráči jeho statistiky z aktuální hry. Vesta také komunikuje s hráčovou zbraní.

\subsection{Zbraň}
Zbraň má jednoduchou úlohu, zajištuje vysílání \zkratka{IR} paprsku, který při zásahu vesty zabije hráče (jde o zabité ve hře ne o skutečné usmrcení).

\subsection{Laser Game systémy na trhu}
