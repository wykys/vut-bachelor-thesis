\section{Laser Game systémy na trhu}
Zjitování informací o systémech na trhu není snadné, výrobci mají na webovích stránkách obvykle jen minimum informací, které vedou na kontaktní formulář pro případného kupce.

\subsection{EVO-5}

EVO-5 je nejpoužívanějším \zkratka{LG} systémem v české republice. Vyvíjí jej norská firma LasserMax a je nasazen například v arénách ... seznam .... Protože je tolik známí, je považován za nepsaný standard a můj systém je jím do značné míry ovlivněn. Po kontaktování výrobce mi poskytnul alespoň ceníky. Pořizovací cena tohoto systému pro vybavení menší arény značná $30~000$~€, což je hlavním důvodem proč se vývojem nového konkurenceschopného systému zabívat.

Naštěstí mi majitel olomoucké \zkratka{LG} arény dovolil systém prohlédnou, tak mohu popsat jak funguje.

\subsubsection{Vesta}
Vesta má na hrudi upevněnou hlavní \zkratka{DPS}. Ta obsahuje řídící mikrokontrolér, komunikační WiFi modul, jeden \zkratka{IR} detektor obklopení čtveřicí \zkratka{RGB} \zkratka{LED} pro identifikaci teamu, znakový display a konektory pro připojení zbraně, a dalších \zkratka{DPS}. Na zádech vesty je upevňěna \zkratka{DPS} obsahující nabíječku, \zkratka{IR} detektor a čtyři \zkratka{RGB} \zkratka{LED}. Na zádech vesty je také upevněn akumulátor, který napájí celou vestu i zbraň. Dále má vesta dvě totožné \zkratka{DPS} na ramenou, každá obsahuje tři \zkratka{IR} detektory a dvě \zkratka{RGB} \zkratka{LED}.

\subsubsection{Zbraň}
Zbraň má ve svích útrobách vysílací $5~\jedn{mm}$ \zkratka{IR} \zkratka{LED} diodu s velmi malím vyzařovacím úhlem. Pro ještě větší zůžení vysílaného paprsku je tato dioda vsazena do $10~\jedn{cm}$ mosazné trubičky, to má za následek zvíšení selektivnosti zbraně. Vesta je schopna detekovat zásah s této zbraně do vzdálenosti $13~\jedn{m}$. Dále zbraň osahuje jeden \zkratka{IR} detektor. Zásah zbraně má stejné následky jako zasažení hráče.Dále zbrna obsahuje a dvě \zkratka{RGB} \zkratka{LED} pro identifikaci teamu a jednu bílou $5~\jedn{mm}$ \zkratka{LED}, která slouží jako svítilna. Dále ve zbrani jsou dva mikrospínače, jeden sloučí jako spouť a dtuhý má programovatlenou funkci a to pujistka střesly nebo rozsvícení svítilny.

\subsubsection{Řídící počítač}
Řídící počítač je založen na klasickém \zkratka{PC} s operačním systémem Microsoft Windows~10, na kterém běří řídící \zkratka{SW}. Operátor jejím prostřednictvím může zapínat a vypínant vesty, nastavovat jména hráčů, přiřazovat je to teamů a zvolit typ hry. Po dokončení hry umožnuje \zkratka{SW} vytisknout výsledky.
