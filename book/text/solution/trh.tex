\section{Laser Game systémy na trhu}
Zjišťování informací o systémech na trhu není snadné, výrobci mají na webových stránkách obvykle jen minimum informací, které vedou ke kontaktnímu formuláři pro případné kupce.

\subsection{EVO-5}

EVO-5 je nejpoužívanějším \zkratka{LG} systémem v České republice. Vyvíjí jej nizozemská firma LaserMaxx a v Česku je nasazen například v arénách v Praze, Brně, Olomouci, Ostravě a Zlíně. Protože je tolik známý, je považován za nepsaný standard a můj systém je jím do značné míry ovlivněn. Po kontaktování mi výrobce poskytl alespoň ceníky. Pořizovací cena tohoto systému pro vybavení menší arény začíná na €~$29~995$, což je hlavním důvodem, proč se vývojem nového konkurenceschopného systému zabývat. Systém EVO-5 umožňuje hru až 65 hráčů. Autor měl možnost si tento systém prohlédnout, následuje jeho rámcový popis:

\paragraph{Vesta}
má na hrudi upevněnou hlavní DPS. Ta obsahuje řídící mikrokontrolér, komunikační \zk{WiFi} modul, jeden IR detektor obklopen čtveřicí RGB LED pro identifikaci týmu, znakový displej a konektory pro připojení zbraně a dalších DPS. Na zádech vesty je upevněn modul obsahující nabíječku, IR detektor a čtyři RGB LED. Na zádech vesty je také upevněn akumulátor, který napájí celou vestu i zbraň. Dále má vesta dvě totožné DPS na ramenou, každá obsahuje tři IR detektory a dvě RGB LED.

\paragraph{Zbraň}
má ve svých útrobách vysílací $5\jedn{mm}$ IR LED diodu s velmi malým vyzařovacím úhlem. Pro ještě větší zúžení vysílaného paprsku je tato dioda vsazena do $10\jedn{cm}$ mosazné trubičky, to má za následek zvýšení směrové selektivity zbraně. LaserMaxx na svých stránkách uvádí, že vesta je schopna detekovat zásah z této zbraně do vzdálenosti $35~\jedn{m}$. Dále zbraň obsahuje jeden IR detektor. Zásah zbraně má stejné následky jako zasažení hráče. Dále zbraň obsahuje a dvě RGB LED pro identifikaci týmu a jednu bílou $5\jedn{mm}$ LED, která slouží jako svítilna. Dále ve zbrani jsou dva mikrospínače, jeden slouží jako spoušť a druhý má programovatelnou funkci a to pojistku střelby nebo rozsvícení svítilny.

\paragraph{Řídící počítač}
je založen na klasickém \zkratka{PC} s operačním systémem Microsoft Windows~10, na kterém běží řídící \zkratka{SW}. Operátor jeho prostřednictvím může zapínat a vypínat vesty, nastavovat jména hráčů, přiřazovat je do týmů a zvolit typ hry. Po dokončení hry umožňuje \zkratka{SW} vytisknout výsledky. V počítači je \zk{WiFi} karta pro komunikaci s vestami.


\subsection{Další systémy}
Další systémy používané v České republice jsou systém REVOLUTION of firmy Delta Strike z Nového Zélandu používaný v jedné brněnské aréně a systém Laserforce používaný jednou pražskou arénou, o kterém se dá najít ještě méně informací. Tyto systémy bohužel neprošly bližším zkoumáním autora.
