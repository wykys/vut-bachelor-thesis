\section{Laser Game systémy na trhu}
Zjišťování informací o systémech na trhu není snadné, výrobci mají na webových stránkách obvykle jen minimum informací, které vedou na kontaktní formulář pro případného kupce.

\subsection{EVO-5}

EVO-5 je nejpoužívanějším \zkratka{LG} systémem v České republice. Vyvíjí jej nizozemská firma LaserMaxx a v Česku je nasazen například v arénách Praze, Brně, Olomouci, Ostravě a Zlíně. Protože je tolik známý, je považován za nepsaný standard a můj systém je jím do značné míry ovlivněn. Po kontaktování mi výrobce poskytnul alespoň ceníky. Pořizovací cena tohoto systému pro vybavení menší arény značná na €~$29~995$, což je hlavním důvodem, proč se vývojem nového konkurenceschopného systému zabývat. Systém EVO-5 umožnuje hru až 65 hráčů.

Naštěstí jsem měl možnost si systém prohlédnou, tak mohu popsat jak funguje.

\subsubsection{Vesta}
Vesta má na hrudi upevněnou hlavní \zkratka{DPS}. Ta obsahuje řídící mikrokontrolér, komunikační \zk{WiFi} modul, jeden \zkratka{IR} detektor obklopen čtveřicí \zkratka{RGB} \zkratka{LED} pro identifikaci teamu, znakový display a konektory pro připojení zbraně, a dalších \zkratka{DPS}. Na zádech vesty je upevňěna \zkratka{DPS} obsahující nabíječku, \zkratka{IR} detektor a čtyři \zkratka{RGB} \zkratka{LED}. Na zádech vesty je také upevněn akumulátor, který napájí celou vestu i zbraň. Dále má vesta dvě totožné \zkratka{DPS} na ramenou, každá obsahuje tři \zkratka{IR} detektory a dvě \zkratka{RGB} \zkratka{LED}.

\subsubsection{Zbraň}
Zbraň má ve svých útrobách vysílací $5~\jedn{mm}$ \zkratka{IR} \zkratka{LED} diodu s velmi malým vyzařovacím úhlem. Pro ještě větší zúžení vysílaného paprsku je tato dioda vsazena do $10~\jedn{cm}$ mosazné trubičky, to má za následek zvýšení selektivnosti zbraně. LaserMaxx na svých stránkách uvádí, že vesta je schopna detekovat zásah z této zbraně do vzdálenosti $35~\jedn{m}$. Dále zbraň obsahuje jeden \zkratka{IR} detektor. Zásah zbraně má stejné následky jako zasažení hráče. Dále zbraň obsahuje a dvě \zkratka{RGB} \zkratka{LED} pro identifikaci teamu a jednu bílou $5~\jedn{mm}$ \zkratka{LED}, která slouží jako svítilna. Dále ve zbrani jsou dva mikrospínače, jeden slouží jako spoušť a druhý má programovatelnou funkci a to pojistku střelby nebo rozsvícení svítilny.

\subsubsection{Řídící počítač}
Řídící počítač je založen na klasickém \zkratka{PC} s operačním systémem Microsoft Windows~10, na kterém běží řídící \zkratka{SW}. Operátor jejím prostřednictvím může zapínat a vypínat vesty, nastavovat jména hráčů, přiřazovat je do teamů a zvolit typ hry. Po dokončení hry umožnuje \zkratka{SW} vytisknout výsledky. V počítači je \zk{WiFi} karta pro komunikaci s vestami.


\subsubsection{Další systémy}
Mezi další systémy používané v české republice patří systém REVOLUTION of firmy Delta Strike z Nového Zélandu používaný v jedné brněnské aréně a systém Laserforce používaný jednou pražskou arénou, o kterém se dá najít ještě méně informací. Bohužel tyto systémy jsem neměl možnost prozkoumat.
