\chapter{Úvod do problematiky Laser Game systému}

\section{Laser Game}
\zkratka{LG}, někdy také nazývána i Laser Tag, je označení pro společenské hry, motivované sci-fi, využívající moderní elektroniku. Hráči získávají body za zásahy \zkratka{IR} paprskem z jejich zbraně do zásahových čidel soupeřů či jiných objektů vybavených zásahovým čidlem. A naopak o své body přicházejí, jestliže se jejich zásahová čidla stanou terčem nepřátelských \zkratka{IR} paprsků. Cílem každého hráče je získat co nejvíce bodů.

Nejčastěji se hra odehrává v místnosti speciálně upravené pro tento účel, označované jako aréna. Stěny, strop a podlaha arény bývají zpravidla pokryty tmavou barvou (černé koberce, koženky a podobné materiály) tak, aby se zabránilo odrazům \zkratka{IR} paprsků. Zdi v arénách jsou často vypolstrovány, aby se snížilo riziko zranění hráčů. Obvykle arény nejsou jen klasické místnosti se čtyřmi stěnami a rovnou podlahou, nýbrž často bývají značně členité, vybavené úkryty a překážkami.

Pro zpestření hry mohou být arény vybaveny i další elektronikou, například optickými závorami, minami či světelnými nebo kouřovými efekty.

\section{Historie Laser Game}
Koncem sedmdesátých a na začátku osmdesátých let vyvíjela firma Lockheed Martin Information Systems pro armádu Spojených států amerických systém, využívající \zkratka{IR} paprsky pro trénink boje nazývaný \zkratka{MILIES}. Systém funguje na stejném principu jako \zkratka{LG}. Voják má na hlavni zbraně \zkratka{IR} vysílač, který reaguje na stisk spouště. Voják je oděn ve speciálním obleku s \zkratka{IR} senzory. Při zásahu vojáka jsou odeslány z obleku informace do řídící jednotky, která vyhodnotí, zda je voják zraněn či virtuálně mrtev - používání tohoto systému je bezpečnější než využití konvenčních zbraní. V roce 1992 vyšla nová verze tohoto systému pojmenovaná MILIES2000, který navíc po virutální smrti vojáka zablokuje jeho další střelbu a akusticky ohlásí zasažení. \zkratka{MILIES} využívá od roku 2002 i Armáda České republiky a také i ostatní země z \zkratka{NATO}.

Prvním komerčně prodávaným zařízením, se kterým je spojován vznik \zkratka{LG}, je hračka Star Trek Electronic Phaser od společnosti South end Electronics, která byla uvedena na trh v roce 1979. Hračka má tvar futuristické zbraně inspirované Star Trekem. Součástí zbraně je signalizační blok, který pomocí \zkratka{LED} a reproduktoru indikuje zásah a \zkratka{IR} vysílač pro palbu.

\section{Laser Game systém}
\zkratka{LG} systém je soustava zahrnující \zkratka{HW}, \zkratka{FW} a \zkratka{SW} umožnující hrát \zkratka{LG}. Tento systém lze tedy chápat jako vybavení nezbytné pro provozování celé arény.

Nejdůležitějšími součástmi \zkratka{LG} systému jsou vesty, zbraně, komunikační směrovač a řídící \zkratka{PC}, \zkratka{LG} systém může být dále doplněn například o optické závory, miny či bomby.

\begin{figure}[H]
    \begin{center}
        \includegraphics[width=\textwidth]{img/lgs}
    \end{center}
    \caption{Znázornění \zkratka{LG} systému}
\end{figure}

\subsection{Řídící počítač}
Řídící počítač má za úkol prostřednictvím směrovače komunikovat s vestami, případně i dalšími zařízeními v síti. Hostuje řídící \zkratka{SW}, pomocí něhož operátor arény ovládá hru. Zajišťuje nakonfigurování vest, v průběhu hry stahuje z vest aktuální údaje a vyhodnocuje pořadí hráčů. Často bývá doplněn tiskárnou, díky níž po vyhodnocení hry vytiskne statistiky z právě odehrané hry.

\subsection{Směrovač}
Zajišťuje směrování v bezdrátové síti - zprostředkovává tedy duplexní spojení mezi vestami a počítačem.

\subsection{Vesta}
Vesta je oděv, který nosí hráči - má dvě hlavní funkce, detekovat zásahy a komunikovat s řídícím počítačem. Další funkcí vesty je barevná identifikace hráčů pomocí \zkratka{RGB} \zkratka{LED}. Vesta může zajišťovat i zvukové efekty, či zobrazovat hráči jeho statistiky z aktuální hry. Vesta také komunikuje s hráčovou zbraní a vyhodnocuje, jestli může daný hráč střílet. Existují řešení, kdy je vesta nahrazena systémem tagů upevněných na oblečení, tato varianta ale není moc oblíbená, vzhledem k tomu, že upevňování tagů je zdlouhavé a během hry hrozí jejich odpadnutí.

\subsection{Zbraň}
Zbraň má jednoduchou úlohu, zajišťuje vysílání \zkratka{IR} paprsku, který při zásahu vesty způsobí virtuální zabití hráče. Také může obsahovat \zkratka{IR} detektor, díky kterému může být signalizováno poškození zbraně jejím zásahem. Dále může být i zbraň vybavena signalizační \zkratka{RGB} \zkratka{LED} pro identifikaci týmu, svítilnou či displejem.
