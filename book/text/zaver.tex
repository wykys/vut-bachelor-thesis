\chapter{Závěr}

Podařilo se mi navrhnout velmi spolehlivý protokol pro přenos dat pomocí infračerveného záření, založený na 16~\jedn{b} CRC, díky tomu je chybovost systému 15~\jedn{ppm}. První verze návrhu se potýkala s problémem zahlcování detektorů shluky nul a následnou chybnou interpretací zkreslených dat.

Druhá verze problémy odstranila návrhem nového robustního komunikačního protokolu. Navržený protokol korektně přenáší i rámce plné nul, protože byla změněna reprezentace logických hodnot v komunikaci. Výhodou tohoto systému je i možnost zjistit výpadek při probíhající komunikaci.

TODO: - plánuji přepsat, přidat výpočty ke zdrojům a některé kapitoly ještě rozšířit
